% -*- mode: latex; mode: visual-line; fill-column: 9999; coding: utf-8 -*-

\section{Additional Data}
\label{sec:supplement}

Figure \ref{fig:MPIwithIO-Bridges} shows performance of the RMSD task on \emph{PSC Bridges}. 

\begin{figure}[!htb]
  \centering
  \begin{subfigure}{.4\textwidth}
    \includegraphics[width=\linewidth]{figures/main-RMSD-t_total-Bridges.pdf}
    \caption{Scaling total}
    \label{fig:MPIscaling-Bridges}
  \end{subfigure}
  \hfill
  \begin{subfigure}{.4\textwidth}
    \includegraphics[width=\linewidth]{figures/main-RMSD-speed_up-Bridges.pdf}
    \caption{Speed-up}
    \label{fig:MPIspeedup-Bridges}
  \end{subfigure}
  \bigskip

  \begin{subfigure}{.45\textwidth}
    \includegraphics[width=\linewidth]{figures/main-RMSD-time_comp_IO_comparison-Bridges.pdf}
    \captionsetup{format=hang}
    \caption{Scaling for different components}
    \label{fig:ScalingComputeIO-Bridges}
  \end{subfigure}
  \hfill
  \begin{subfigure} {.5\textwidth}
    \includegraphics[width=\linewidth]{figures/main-RMSD-BarPlot-rank-comparison_72_4-Bridges.pdf}
    \captionsetup{format=hang}
    \caption{Time comparison on different parts of the calculations per MPI rank (example)}
    \label{fig:MPIranks-Bridges}
  \end{subfigure}
  \caption{\emph{PSC Bridges}: Performance of the RMSD task.
    Results are communicated back to rank 0.
    Five independent repeats were performed to collect statistics.
    (a-c) The error bars show standard deviation with respect to the mean.
    In serial, there is no communication and hence no data point is shown for $N=1$ in (c).
    (d) Compute \tcomp, read I/O \tIO, communication \tcomm, ending the for loop $t_{\text{end\_loop}}$, opening the trajectory $t_{\text{opening\_trajectory}}$, and overheads $t_{\text{overhead1}}$, $t_{\text{overhead2}}$ per MPI rank; see Table \ref{tab:notation} for definitions.
    These are data from one run of the five repeats.
    MPI ranks 0 and 70 are stragglers.
  }
\label{fig:MPIwithIO-Bridges}
\end{figure} 



Figure \ref{fig:MPIwithIO-SuperMIC} shows performance of the RMSD task on \emph{LSU SuperMIC}. 

\begin{figure}[!htb]
  \centering
  \begin{subfigure}{.4\textwidth}
    \includegraphics[width=\linewidth]{figures/main-RMSD-t_total-SuperMIC.pdf}
    \caption{Scaling total}
    \label{fig:MPIscaling-SuperMIC}
  \end{subfigure}
  \hfill
  \begin{subfigure}{.4\textwidth}
    \includegraphics[width=\linewidth]{figures/main-RMSD-speed_up-SuperMIC.pdf}
    \caption{Speed-up}
    \label{fig:MPIspeedup-SuperMIC}
  \end{subfigure}
  \bigskip

  \begin{subfigure}{.45\textwidth}
    \includegraphics[width=\linewidth]{figures/main-RMSD-time_comp_IO_comparison-SuperMIC.pdf}
    \captionsetup{format=hang}
    \caption{Scaling for different components}
    \label{fig:ScalingComputeIO-SuperMIC}
  \end{subfigure}
  \hfill
  \begin{subfigure} {.5\textwidth}
    \includegraphics[width=\linewidth]{figures/main-RMSD-BarPlot-rank-comparison_80_5-SuperMIC.pdf}
    \captionsetup{format=hang}
    \caption{Time comparison on different parts of the calculations per MPI rank (example)}
    \label{fig:MPIranks-SuperMIC}
  \end{subfigure}
  \caption{\emph{LSU SuperMIC}: Performance of the RMSD task with MPI.
    Results are communicated back to rank 0.
    Five independent repeats were performed to collect statistics.
    (a-c) The error bars show standard deviation with respect to mean.
    In serial, there is no communication and hence the data points for $N=1$ are not shown in (c).
    (d) Compute \tcomp, read I/O \tIO, communication \tcomm, ending the for loop $t_{\text{end\_loop}}$,  opening the trajectory $t_{\text{opening\_trajectory}}$, and overheads $t_{\text{overhead1}}$, $t_{\text{overhead2}}$ per MPI rank; see Table \ref{tab:notation} for definitions.
    These are data from one run of the five repeats.
  }
  \label{fig:MPIwithIO-SuperMIC}
\end{figure} 


Figure \ref{fig:comparison_efficiency_clusters} shows comparison of the parallel efficiency of the RMSD task between different test cases on \emph{SDSC Comet}, \emph{PSC Bridges}, and \emph{LSU SuperMIC} when reading from a HDF5 file.

\begin{figure}[!htb]
  \centering
  \begin{subfigure}{.3\textwidth}
    \includegraphics[width=\linewidth]{figures/Comparison_Efficiency_all_Comet.pdf}
    \caption{\emph{SDSC Comet}}
    \label{fig:comparison_efficiency}
  \end{subfigure}
  \hfill
  \begin{subfigure}{.35\textwidth}
    \includegraphics[width=\linewidth]{figures/Comparison_Efficiency_all_Bridges.pdf}
    \caption{\emph{PSC Bridges}}
    \label{fig:comparison_efficiency_Bridges}
  \end{subfigure}
  \hfill
  \begin{subfigure}{.3\textwidth}
    \includegraphics[width=\linewidth]{figures/Comparison_Efficiency_all_SuperMIC.pdf}
    \caption{\emph{LSU SuperMIC}}
    \label{fig:comparison_efficiency_SuperMIC}
  \end{subfigure}
  \caption{Comparison of the parallel efficiency between different test cases on (a) \emph{SDSC Comet} (data for ``MPI Parallel IO'' are only shown up to 192 cores for better comparison across different scenarios, see Fig.~\protect\ref{fig:MPIspeedup-hdf5} for equivalent scaling data up to 384 cores), (b) \emph{PSC Bridges}, and (c) \emph{LSU SuperMIC}.
    Five repeats were performed to collect statistics and error bars show standard deviation with respect to mean.}
  \label{fig:comparison_efficiency_clusters}
\end{figure} 

