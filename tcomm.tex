\label{sec:tcomm}

In addition to the compute to I/O ratio \RcompIO discussed in Section \ref{sec:bound} we defined another performance parameter called the compute to communication ratio $\Rcompcomm$ (Eq.~\ref{eq:Compute-comm}).
In Section \ref{sec:I/O}, we overcame the I/O effect by splitting the trajectory, but scaling remained far from ideal when MPI communication was used (instead of  Global Arrays).
This is because the task remained communication bound (Figure \ref{fig:MPIwithIO-split}), i.e, 
\begin{gather*}
  \Rcompcomm \ll 1.
\end{gather*}

Figure \ref{fig:tcom_tcomm_effect} shows the relationship of performance with $\Rcompcomm$ ratio.
When the $\Rcompcomm$ ratio is higher (Figure \ref{fig:tcomp_tcomm_ratio}), performance is better (Figure \ref{fig:S1_tcomp_tcomm_effect}) even if communication time is larger (Figure \ref{fig:Comm_time_tcomp_tcomm_effect}).
Although, we still observed stragglers due to communication at larger $\Rcompcomm$ ratios ($70\times$ RMSD and $100\times$ RMSD), their effect on performance remained modest because the overall performance was dominated by the compute load. 
Evidently, if overall performance is dominated by a component such as compute that scales well, then performance problems with components such as communication will be masked and overall acceptable performance can still be achieved (Figures \ref{fig:S1_tcomp_tcomm_effect} and \ref{fig:tcomp_tcomm_ratio}).

Communication is usually not problematic within one node because of the shared memory environment (for less than 24 processes (single compute node on \emph{SDSC Comet}) the scaling is good and $\Rcompcomm \gg 1$ for all RMSD loads) (Figures \ref{fig:S1_tcomp_tcomm_effect} and \ref{fig:tcomp_tcomm_ratio}).
However, beyond a single compute node, scaling appears to get better as the $\Rcompcomm$ ratio increases and communication overhead becomes less dominant (Figures \ref{fig:S1_tcomp_tcomm_effect} and \ref{fig:tcomp_tcomm_ratio}, 24-72 cores represent multiple compute nodes on \emph{SDSC Comet}).

\begin{figure}[!htb]
  \centering
  \begin{subfigure} {.33\textwidth}
    \includegraphics[width=\linewidth]{figures/Compute_to_IO_ratio_on_performance_2d_v17.pdf}
    \caption{Speed-Up $S(N)$}
    \label{fig:S1_tcomp_tcomm_effect}
  \end{subfigure}
  \hfill
  \begin{subfigure}{.3\textwidth}
    \includegraphics[width=\linewidth]{figures/Compute_to_comm_ratio_on_performance_v17.pdf}
    \captionsetup{format=hang}
    \caption{Compute to communication ratio \Rcompcomm}
    \label{fig:tcomp_tcomm_ratio}
  \end{subfigure}
  \hfill
  \begin{subfigure}{.33\textwidth}
    \includegraphics[width=\linewidth]{figures/comm_comparison_different_RMSD_overload.pdf}
    \caption{Communication time \tcomm}
    \label{fig:Comm_time_tcomp_tcomm_effect}
  \end{subfigure}
  \caption{Effect of the ratio of compute to communication time \Rcompcomm on scaling performance on \emph{SDSC Comet}.
    (a) Scaling for different computational workloads.
    (b) Change in \Rcompcomm with the number of processes $N$ for different workloads. 
    (c) Comparison of communication time for different RMSD workloads.
    Five repeats were performed to collect statistics and error bars show standard deviation with respect to mean.}
  \label{fig:tcom_tcomm_effect}
\end{figure}

%%% Local Variables:
%%% mode: latex
%%% TeX-master: t
%%% End:
