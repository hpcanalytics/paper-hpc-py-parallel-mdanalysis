% -*- mode: latex; mode: visual-line; fill-column: 9999; coding: utf-8 -*-

\section{Effect of \Rcompcomm on Performance}
\label{sec:tcomm}

In Section \ref{sec:I/O}, we improved scaling limitations due to read I/O by splitting the trajectory, but scaling remained far from ideal when MPI communication was used; somewhat surprisingly, using \emph{Global Arrays} lead to better scaling (see Section \ref{Global-Array}) because the effective communication cost was reduced \mknote{I really disagree we this sentence. I do not understand why the results of the GA is surprising for every one. I have tried to explain why GA lead to better perforance. What is not clear to you?}.
Although we were not able to identify the reason for the better performance of \emph{Global Arrays} (it still uses MPI as a communicator), the results motivated an analysis in terms of the communication costs.
In addition to the compute to I/O ratio \RcompIO discussed in Section \ref{sec:bound} we defined another performance parameter called the compute to communication ratio $\Rcompcomm$ (Eq.~\ref{eq:Compute-comm}).

We analyzed the data for variable workloads (see Section \ref{sec:bound}) in terms of the $\Rcompcomm$ ratio.
The performance clearly depended on the $\Rcompcomm$ ratio (Figure \ref{fig:tcom_tcomm_effect}).
Performance improved with increasing $\Rcompcomm$ ratios (Figure \ref{fig:tcomp_tcomm_ratio} and \ref{fig:S1_tcomp_tcomm_effect}) even if the communication time was larger (Figure \ref{fig:Comm_time_tcomp_tcomm_effect}).
Although we still observed stragglers due to communication at larger $\Rcompcomm$ ratios ($70\times$ RMSD and $100\times$ RMSD), their effect on performance remained modest because the overall performance was dominated by the compute load. 
Evidently, as long as overall performance is dominated by a component such as compute that scales well, then performance problems with components such as communication will be masked and overall acceptable performance can still be achieved (Figures \ref{fig:S1_tcomp_tcomm_effect} and \ref{fig:tcomp_tcomm_ratio}).

Communication was usually not problematic within one node because of the shared memory environment.
For less than 24 processes, i.e., a single compute node on \emph{SDSC Comet}, the scaling was good and $\Rcompcomm \gg 1$ for all RMSD loads (Figures \ref{fig:S1_tcomp_tcomm_effect} and \ref{fig:tcomp_tcomm_ratio}).
However, beyond a single compute node ($>$ 24 cores), scaling appeared to improve with increasing $\Rcompcomm$ ratio while the communication overhead decreased in importance (Figures \ref{fig:S1_tcomp_tcomm_effect} and \ref{fig:tcomp_tcomm_ratio}).

\begin{figure}[!htb]
  \centering
  \begin{subfigure} {.33\textwidth}
    \includegraphics[width=\linewidth]{figures/Compute_to_IO_ratio_on_performance_2d_v17.pdf}
    \caption{Speed-Up $S(N)$}
    \label{fig:S1_tcomp_tcomm_effect}
  \end{subfigure}
  \hfill
  \begin{subfigure}{.3\textwidth}
    \includegraphics[width=\linewidth]{figures/Compute_to_comm_ratio_on_performance_v17.pdf}
    \captionsetup{format=hang}
    \caption{Compute to communication ratio \Rcompcomm}
    \label{fig:tcomp_tcomm_ratio}
  \end{subfigure}
  \hfill
  \begin{subfigure}{.33\textwidth}
    \includegraphics[width=\linewidth]{figures/comm_comparison_different_RMSD_overload.pdf}
    \caption{Communication time \tcomm}
    \label{fig:Comm_time_tcomp_tcomm_effect}
  \end{subfigure}
  \caption{Effect of the ratio of compute to communication time \Rcompcomm on scaling performance on \emph{SDSC Comet}.
    (a) Scaling for different computational workloads. (Same as Figure~\protect\ref{fig:S1_tcomp_tIO_effect}.)
    (b) Change in \Rcompcomm with the number of processes $N$ for different workloads. 
    (c) Comparison of communication time for different RMSD workloads.
    Five repeats were performed to collect statistics and error bars show standard deviation with respect to mean.}
  \label{fig:tcom_tcomm_effect}
\end{figure}

%%% Local Variables:
%%% mode: latex
%%% TeX-master: t
%%% End:
