\label{background}
\obnote{The lit. review is required for a proper scholarly treatment of the subject. 
Background/Related Work should contain a literature review of other approaches for analyzing MD trajectories on HPC: 
at a minimum: VMD, HiMach, cpptraj (and pytraj), possibly mdtraj ? read the papers/websites and check for benchmarks. Look for others 
(Giannis already mentioned some in his paper and/or the workshop draft).}

CPPTraj~\cite{cpptraj-2013} offers three levels of parallelization. The two are through MPI and the third thourh OpenMP.
The MPI types of parallelization CPPTraj supports are show in Figure~\ref{fig:cpptraj_arch}. In more detail,
CCPTraj allows the parallel read between frames of the same trajectory or ensemble members of
the same trajectory. When it is used to analyze a single trajectory, all frames of the trajectory
are equally distributed over the number of MPI process that are used. Each process reads the frames
that are assigned to it, executes and writes the results of all processes to the same output file.
When ensemble mode is used, each ensemble member is assigned to an MPI process. As a consequence,
there have to be as many MPI processes as ensemble members. The user has the ability to increase 
CPPTraj's throughput by assigning more than one MPI processes per ensemble member. Each ensemble 
member is divided further the same way as a single trajectory.
\begin{figure}[t]
		\begin{subfigure}{.5\textwidth}
		\centering
		\includegraphics[width=.95\textwidth]{figures/CPPTrajExecutionSchematicSingleTrajectory.pdf}
	\end{subfigure}
	\begin{subfigure}{.5\textwidth}
		\centering
		\includegraphics[width=.95\linewidth]{figures/CPPTrajExecutionSchematicEnsembleTrajectories.pdf}
	\end{subfigure}
	\caption{CPPTraj MPI modes of execution. The right figure shows the case where a single trajectory is
		given. The left figure shows the case where an ensemble of trajectories are given for analysis}
	\label{fig:cpptraj_arch}
\end{figure}

HiMach~\cite{himach-2008} was developed by D.E.Shaw Research group to provide a parallel analysis
framework for molecular dynamics simulations. HiMach extends Google's MapReduce
to provide a scalable API for MD trajectory analysis. HiMach API provides a series
of Python classes that are used to define trajectories, do per frame data acquisition
(Map class) and cross-frame analysis (Reduce class). After the user has defined all 
the above, HiMach's runtime is responsible to parallelize and distribute the Map and 
Reduce classes to the assigned cores. Data transfers between Map and Reduce phases are
done through a communication protocol created specifically for HiMach and it is transparent
from the user.