\newif\ifdraft
\drafttrue

\usepackage{breakurl}
\usepackage{url}
\usepackage{adjustbox}
\usepackage{amsmath}
\renewcommand{\ttdefault}{pcr}
\usepackage{tabularx}
\usepackage{textcomp}
%\usepackage{natbib}
\setcitestyle{square, comma, numbers,sort&compress, super}
\usepackage{hypernat}
%\bibpunct{}{}{,}{s}{,}{,}
%% Remove brackets from numbering in List of References
\makeatletter
\renewcommand{\@biblabel}[1]{\quad#1.}
\makeatother
%% no space
\setlength{\bibsep}{0pt}

%% use \carriagereturn from dingbats 
% fix LaTeX Error: Command \checkmark already defined.
\let\checkmark\undefined
\usepackage{dingbat}
\usepackage{color}
\definecolor{mauve}{rgb}{0.58,0,0.82}
\definecolor{gray}{rgb}{0.5,0.5,0.5}
\renewcommand{\ttdefault}{pcr}
\usepackage{listings}

% Python style for highlighting
\newcommand\pythonstyle{\lstset{
    language=Python,
    basicstyle=\scriptsize\ttfamily,
    otherkeywords={self},             % Add keywords here
    keywordstyle=\bfseries\color{blue},
    emph={MyClass,__init__},          % Custom highlighting
    emphstyle=\ttfamily\bfseries\color{deepred},    % Custom highlighting style
    stringstyle=\color{mauve},
    commentstyle=\color{gray}\textit,      
    frame=tb,                         % Any extra options here
    showstringspaces=false,           %
    breaklines=true,
    prebreak=\carriagereturn,
    numberstyle=\tiny\color{gray},
    numbers=left,
    stepnumber=1
}}

% Python environment
\lstnewenvironment{python}[1][]
{
\pythonstyle
\lstset{#1}
}
{}

% Python for inline
\newcommand{\pythoninline[1]}{{\pythonstyle\lstinline!#1!}}

\usepackage{xspace}
\newcommand{\NaH}{Na$^{+}$/H$^{+}$\xspace}
\newcommand{\Na}{Na$^{+}$\xspace}
\newcommand{\Hp}{H$^{+}$\xspace}
\newcommand{\pKa}{\ensuremath{\mathrm{p}K_{\mathrm{a}}}\xspace}
\newcommand{\pK}{\ensuremath{\mathrm{p}K}}
\newcommand{\kon}{\ensuremath{k_{\text{on}}}}
\newcommand{\koff}{\ensuremath{k_{\text{off}}}}
\newcommand{\Aim}[1]{\textsc{Aim #1}\xspace}

\newcommand{\AIM}[2]{\paragraph{\Aim{#1}: #2}}

\newcommand{\PDB}[1]{PDB ID \textsc{#1}\xspace}


\newcommand{\nuvec}[1]{\boldsymbol{\nu}_{#1}}

\newcommand{\mus}{\ensuremath{\mu\text{s}}\xspace}

\newcommand{\package}[1]{\textsl{#1}}
\newcommand{\class}[1]{\textrm{#1}}

\newcommand{\tcomp}{\ensuremath{{t}_{\text{comp}}}\xspace}
\newcommand{\tIO}{\ensuremath{{t}_{\text{I/O}}}\xspace}
\newcommand{\tcomm}{\ensuremath{t_{\text{comm}}}\xspace}
\newcommand{\toverhead}{\ensuremath{t_{\text{overhead}}}\xspace}
%-------------------------------------------------------------------------------------------------------
\ifdraft
%\usepackage{textcomp}
\usepackage{xcolor}
\definecolor{ocolor}{rgb}{1,0,0.4}
\newcommand{\terminology}[1]{ {\textcolor{red} {(Terminology used: \textbf{#1}) }}}
\newcommand{\owave}[1]{ {\cyanuwave{#1}}}
\newcommand{\jwave}[1]{ {\reduwave{#1}}}
\newcommand{\alwave}[1]{ {\blueuwave{#1}}}
\newcommand{\jhanote}[1]{ {\textcolor{red} { ***shantenu: #1 }}}
\newcommand{\todo}[1]{ {\textcolor{brown} { TODO #1 }}}
\newcommand{\obnote}[1]{ {\textcolor{cyan} { ***oliver: #1 }}}
\definecolor{orange}{rgb}{1,.5,0}
\definecolor{dandelion}{cmyk}{0,0.29,0.84,0}
\newcommand{\mknote}[1]{ {\textcolor{blue} { ***mahzad: #1 }}}
\newcommand{\gpnote}[1]{{\textcolor{green} {***giannis: #1}}}
\newcommand{\note}[1]{ {\textcolor{magenta} { ***Note: #1 }}}
\else
\newcommand{\terminology}[1]{}
\newcommand{\jwave}[1]{#1}
\newcommand{\alnote}[1]{}
\newcommand{\mknote}[1]{}
\newcommand{\obnote}[1]{}
\newcommand{\rthreenote}[1]{}
\newcommand{\rfournote}[1]{}
\newcommand{\todo}[1]{}
\newcommand{\jhanote}[1]{}
\newcommand{\rtwonote}[1]{}
\newcommand{\gpnote}[1]{}
\newcommand{\note}[1]{}
\fi

\newcommand{\yarn}{YARN\xspace}
\newcommand{\rp}{RADICAL-Pilot\xspace}
\newcommand{\cloud}{cloud\xspace}
\newcommand{\clouds}{clouds\xspace}
\newcommand{\pilot}{Pilot\xspace}
\newcommand{\pilots}{Pilots\xspace}
\newcommand{\pilotjob}{Pilot-Job\xspace}
\newcommand{\pilotjobs}{Pilot-Jobs\xspace}
\newcommand{\pilotcompute}{Pilot-Compute\xspace}
\newcommand{\pilotcomputedescription}{Pilot-Compute Description\xspace}
\newcommand{\pilotdescription}{Pilot-Description\xspace}
\newcommand{\pilotcomputes}{Pilot-Computes\xspace}
\newcommand{\pilotdata}{Pilot-Data\xspace}
\newcommand{\pilotdatainmem}{Pilot-Data Memory\xspace}
\newcommand{\pilotdatadescription}{Pilot-Data Description\xspace}
\newcommand{\pilotdataservice}{Pilot-Data Service\xspace}
\newcommand{\pilotcomputeservice}{Pilot-Compute Service\xspace}
\newcommand{\computedataservice}{Compute-Data Service\xspace}
\newcommand{\computedatamanager}{Compute-Data Manager\xspace}
\newcommand{\computeunitdescription}{Compute-Unit Description\xspace}
\newcommand{\dataunitdescription}{Data-Unit Description\xspace}
\newcommand{\pilotmapreduce}{PilotMapReduce\xspace}
\newcommand{\mrmg}{MR-Manager\xspace}
\newcommand{\pstar}{P*\xspace}
\newcommand{\pd}{PD\xspace}
\newcommand{\pc}{PC\xspace}
\newcommand{\pcs}{PCs\xspace}
\newcommand{\pj}{PJ\xspace}
\newcommand{\pjs}{PJs\xspace}
\newcommand{\pds}{Pilot Data Service\xspace}
\newcommand{\computeunit}{Compute-Unit\xspace}
\newcommand{\computeunits}{Compute-Units\xspace}
\newcommand{\dataunit}{Data-Unit\xspace}
\newcommand{\dataunits}{Data-Units\xspace}
\newcommand{\du}{DU\xspace}
\newcommand{\dus}{DUs\xspace}
\newcommand{\dud}{DUD\xspace}
\newcommand{\cu}{CU\xspace}
\newcommand{\cus}{CUs\xspace}
\newcommand{\cud}{CUD\xspace}
\newcommand{\su}{SU\xspace}
\newcommand{\sus}{SUs\xspace}
\newcommand{\schedulableunit}{Schedulable Unit\xspace}
\newcommand{\schedulableunits}{Schedulable Units\xspace}
\newcommand{\cc}{c\&c\xspace}
\newcommand{\CC}{C\&C\xspace}

\newcommand{\numrep}{8 }
\newcommand{\samplenum}{4 }
\newcommand{\tmax}{$T_{max}$ }
\newcommand{\tc}{$T_{C}$ }
\newcommand{\tcnsp}{$T_{C}$}
\newcommand{\bj}{BigJob\xspace}
\newcommand{\irods}{iRODS\xspace}

\newcommand{\I}[1]{\textit{#1}\xspace}
\newcommand{\B}[1]{\textbf{#1}\xspace}
\newcommand{\T}[1]{\texttt{#1}\xspace}
%\newcommand{\C}[1]{\textsc{#1}\xspace}

% define a new float, with style `ruled`
\floatstyle{ruled}
\newfloat{scheme}{htbp}{lok}
\floatname{scheme}{Scheme}

%\lstdefinestyle{myListing}{
%    frame=single,
%    backgroundcolor=\color{listinggray},
    %float=t,
%    language=C,
%    basicstyle=\ttfamily \footnotesize,
%    breakautoindent=true,
%    breaklines=true
%    tabsize=2,
%    captionpos=b,
%    aboveskip=0em,
%    belowskip=-2em,
    %numbers=left,
    %numberstyle=\tiny
%}

%\lstdefinestyle{myPythonListing}{
%    frame=single,
%    backgroundcolor=\color{listinggray},
    %float=t,
%    language=Python,
%    basicstyle=\ttfamily \scriptsize,
%    breakautoindent=true,
%    breaklines=true
%    tabsize=2,
%    captionpos=b,
    %numbers=left,
    %numberstyle=\tiny
%}


\ifpdf
\DeclareGraphicsExtensions{.pdf, .jpg, .tif}
\else
\DeclareGraphicsExtensions{.ps,  .eps, .jpg}
\fi

\tolerance=1000
\hyphenpenalty=10

%\usepackage{listings}
%\usepackage{paralist}
% Python environment
%\lstnewenvironment{code}[1][]%
%{
%    \noindent
    %\minipage{0.98 \linewidth}
%    \minipage{1.0 \linewidth}
%    \vspace{0.5\baselineskip}
%   \lstset{
%        language=Python,
%       frame=single,
%        captionpos=b,
%        stringstyle=\ttfamily,
%        basicstyle=\scriptsize\ttfamily,
%        showstringspaces=false,#1}
%}
%{\endminipage}

\defaultleftmargin{1em}{}{}{}