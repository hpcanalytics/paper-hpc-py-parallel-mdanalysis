% -*- mode: latex; mode: visual-line; fill-column: 9999; coding: utf-8 -*-

\subsection{Effect of \RcompIO on Scaling Performance}
\label{sec:increasedworkload}

For an $X$-fold increase in workload, we expected the workload for the computation to scale with $X$ as $\tcomp(X) =  N_{\text{frames}}^{\text{total}} X \overline{\tcomp^{\text{frame}}}$ while the read I/O workload $\tIO(X) = N_{\text{frames}}^{\text{total}} \overline{\tIO^{\text{frame}}}$ (number of frames times the average time to read a frame) should remain independent of $X$.
Therefore, the ratio for any $X$ should be $\RcompIO(X) = \tcomp(X)/\tIO(X) = X \RcompIO(X=1)$, i.e.,  $\RcompIO$ should just linearly scale with the workload factor $X$.
The measured $\RcompIO$ ratios of 11, 19, 27 for the increased computational workloads agreed with this theoretical analysis, as shown in Table \ref{tab:load-ratio}.

\begin{SCtable}[1.0][!htb]
\centering
\caption[Change in load-ratio with RMSD workload]{Change in $\RcompIO$ ratio with change in the RMSD workload $X$.
  The RMSD workload was artificially increased in order to examine the effect of compute to I/O ratio on the performance.
  The reported compute and I/O time were measured based on the serial version using one core.
  The theoretical $\RcompIO$ (see text) is provided for comparison.}
\label{tab:load-ratio}
\begin{tabular}{rrrrr}
  \toprule
  \bfseries\thead{Workload $X$} &  \bfseries\thead{$\tcomp$ (s)} &  \bfseries\thead{$\tIO$ (s)}
  & \multicolumn{2}{c}{\bfseries\thead{$\RcompIO$}}\\
  & & & \thead{measured} & \thead{theoretical}\\
  \midrule
    $1\times$   &   226 & 791 &  0.29 &   \\  
    $40\times$  &  8655 & 791 & 11   & 11\\    
    $70\times$  & 15148 & 791 & 19   & 20\\  
    $100\times$ & 21639 & 791 & 27   & 29\\  
  \bottomrule
\end{tabular}
\end{SCtable}

\begin{figure}[!htb]
  \centering
  \begin{subfigure} {.3\textwidth}
    \includegraphics[width=\linewidth]{figures/Compute_to_IO_ratio_on_performance_2d_v17.pdf}
    \caption{Speed-Up}
    \label{fig:S1_tcomp_tIO_effect}
  \end{subfigure}
  \hfill
  \begin{subfigure}{.3\textwidth}
    \includegraphics[width=\linewidth]{figures/Compute_to_IO_ratio_on_performance_2d_2_v17.pdf}
    \caption{Speed-Up}
    \label{fig:S2_tcomp_tIO_effect}
  \end{subfigure}
  \hfill
  \begin{subfigure}{.3\textwidth}
    \includegraphics[width=\linewidth]{figures/Compute_to_IO_ratio_on_performance_2d_3_v17.pdf}
    \caption{Efficiency}
    \label{fig:E_tcomp_tIO_effect}
  \end{subfigure}
  \caption{Effect of $\RcompIO$ ratio on performance of the RMSD task on \emph{SDSC Comet}. We tested performance for $\RcompIO$ ratios of 0.3, 11, 19, 27, which correspond to $1\times$ RMSD, $40\times$ RMSD, $70\times$ RMSD, and $100\times$ RMSD respectively.
    (a) Effect of $\RcompIO$ on the speed-up.
    (b) Change in speed-up with respect to $\RcompIO$ for different processor counts.
    (c) Change in the efficiency with respect to $\RcompIO$ for different processor counts.}
  \label{fig:tcomp_tIO_effect}
\end{figure}

We performed the experiments with increased workload to measure the effect of the $\RcompIO$ ratio (Eq.~\ref{eq:Compute-IO}) on performance (Figure~\ref{fig:tcomp_tIO_effect}).
The strong scaling performance as measured by the speed-up $S(N)$ improved with increasing $\RcompIO$ ratio (Figure \ref{fig:S1_tcomp_tIO_effect}).
The calculations consistently showed better scaling performance to larger numbers of cores for higher $\RcompIO$ ratios, e.g., $N=56$ cores for the $70\times$ RMSD task. 
The speed-up and efficiency approached their ideal value for each processor count with increasing $\RcompIO$ ratio (Figures \ref{fig:S2_tcomp_tIO_effect} and \ref{fig:E_tcomp_tIO_effect}).
Even for moderately compute-bound workloads, such as the $40\times$ and $70\times$ RMSD tasks, increasing the computational workload over I/O reduced the impact of stragglers even though they still contributed to large variations in timing across different ranks and thus irregular scaling.



%%% Local Variables:
%%% mode: latex
%%% TeX-master: "main"
%%% End:
