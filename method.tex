\label{methods}
The test system contained the protein adenylate kinase with 214 amino acid residues and 3341 atoms in total \citep{Seyler:2014il}. 
The trajectory was in Gromacs XTC format trajectory (``600x'' in \citet{Khoshlessan:2017ab}) with a size of about 30 GB and 2,512,200
time frames (corresponding to 602.4~µs simulated time) which represents a typical medium per-frame size but is very long for
current standards.

The experiments were executed on the XSEDE Supercomputers: \emph{SDSC Comet}. 
SDSC Comet is a 2.7 PFlop/s cluster with 6,400 nodes in total.
The standard compute nodes consist of Intel Xeon E5-2680v3 processors, 128 GB DDR4 DRAM (64 GB per socket), and 320 GB of SSD local scratch memory. 
The large memory nodes contain 1.5 TB of DRAM and four Haswell processors each and the network topology is 56 Gbps FDR Infini-Band.
All the experiments in the present study are performed using compute nodes.

\subsection*{Timing observables}
We model MPI performance based on the RMSD algorithm (\ref{alg:RMSD}) and Dihedral Featurization algorithm (\ref{alg:Dihedral}). 
The notation for our models is summarized in Table \ref{tab:notation}.
We directly measured inside our code (in the function \texttt{block\_rmsd()}) the ``compute'' time per
trajectory frame to perform the computation \tcomp (\texttt{t\_comp} in the code) and the ``I/O'' time for
ingesting the data from the file system into memory, \tIO (\texttt{t\_IO}). 
$t_{comp\_final}$ is the summation of ``compute'' time per frame and $t_{IO\_final}$ is the summation of ``I/O'' time per frame for all the frames assigned to each rank. 
$t_{end\_loop}$ is the time delay between the end of the last iteration and exiting the for loop.
$t_{opening\_trajectory}$ is the time for where data structures are initialized and topology and trajectory files are opened.

Outside the \texttt{block\_rmsd()} function, relevant probs were taken and stored as variables \texttt{start4},
\texttt{start5}, and \texttt{start6}, which we will abbreviate in the following as $t_{4}$ (after setting up the problem
and before entering \texttt{block\_rmsd()} to \emph{ingest} and \emph{compute}), $ t_{5}$, (before communicating back results with
\texttt{MPI.COMM\_WORLD.Gather()}), and $t_{6}$ (after all work has been done but before timing statistics are computed).  
The total time (for all frames) spent in \texttt{block\_rmsd()} is $t_{\text{RMSD}} = t_{5} - t_{4}$. 
The ``Shuffle'' time to gather (``reduce'') all data from all processor ranks is $\tcomm = t_{6} - t_{5}$.

There are parts of the code in \texttt{block\_rmsd()} that are not covered by the detailed timing information of \tcomp and \tIO. 
To measure the un-accounted time we define the ``overheads''.
$t_{Overhead1}$ and $t_{Overhead2}$ are the overhead of the calculations and they should be ideally very small.  
The total time to completion for a single process on $N$ cores is $t_{N}$, which is mathematically equivalent to
$t_{N} \equiv t_{RMSD} + \tcomm$.

We also recorded the total time to solution $t_{\text{total}}(N)$ with $N$ MPI processes on $N$ cores (which is effectively
$t_{\text{total}}(N) \approx \max t_{N}$). 
Strong scaling was quantified by calculating the speed-up relative to performance on a single core (using MPI).

\begin{gather}
  \label{eq:speedup}
  S = \frac{t_{\text{total}}(N)}{t_{\text{total}}(1)}
\end{gather}

\begin{table}
\centering
\begin{tabular}{c c}
  \toprule
            \thead{Item} & \thead{Definition}\\
  \midrule
    $t_{end\_loop}$ & $t_{3}-t_{1}$\\
    $t_{opening\_trajectory}$ &  $t_{0}-t_{00}$ \\
    $t_{comp\_final}$ &  $np.sum(t_{comp})$\\
    $t_{IO\_final}$ & $np.sum(t_{IO})$\\
    $t_{all\_frame}$ & $t_{3}-t_{0}$  \\
    $t_{RMSD}$ &  $t_{5}-t_{4}$ \\
    $t_{Communication_{MPI}}$ &  $t_{6}-t_{5}$  \\
    $t_{Communication_{GA}}$ &  $(t_{6}-t_{5})$+$(t_{7}-t_{6})$  \\
    $t_{Overhead1}$ & $t_{all\_frame}-t_{IO\_final}-t_{comp\_final}-t_{end\_loop}$  \\
    $t_{Overhead2}$ & $t_{RMSD}-t_{all\_frame}-t_{opening\_trajectory}$  \\
    $t_{N}$ & $t_{RMSD}+t_{Communication}$ \\
    $t_{total}$ & $\max t_{N}$ \\
  \bottomrule
\end{tabular}
\caption[Notation of our performance modeling]
{Notation of our performance modeling (relevant probes in the codes are stored as variables \texttt{start*},
which we will abbreviate in here as $t_{*}$)}
\label{tab:notation}
\end{table}





