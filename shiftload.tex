% -*- mode: latex; mode: visual-line; fill-column: 9999; coding: utf-8 -*-

\section{Effect of Increasing the Computational Load on Scaling
  Performance}
\label{sec:shiftload}

We quantified the effect of increasing the computational load of the analysis task on scaling performance in order to obtain practical insights into the question under which circumstances the simple single-trajectory split-apply-combine MPI-based parallelization approach might be feasible without any other considerations such as optimization of the file read I/O.
To this end, we quantified strong scaling performance as a function of the compute-to-I/O ratio \RcompIO (Eq.~\ref{eq:Compute-IO}) and the compute-to-communication ratio \Rcompcomm (Eq.~\ref{eq:Compute-comm}).

To measure the effect of an increased compute load on performance while leaving other parameters the same, we artificially increased the computational load by repeating the same RMSD calculation (line 10, algorithm \ref{alg:RMSD}) 40, 70 and 100 times in a loop, resulting in forty-fold (``$40\times$''), seventy-fold (``$70\times$''), and one hundred-fold (``$100\times$'') load increases.


% -*- mode: latex; mode: visual-line; fill-column: 9999; coding: utf-8 -*-

\subsection{Effect of \RcompIO on Scaling Performance}
\label{sec:increasedworkload}

For an $X$-fold increase in workload, we expected the workload for the computation to scale with $X$ as $\tcomp(X) =  N_{\text{frames}}^{\text{total}} X \overline{\tcomp^{\text{frame}}}$ while the read I/O workload $\tIO(X) = N_{\text{frames}}^{\text{total}} \overline{\tIO^{\text{frame}}}$ (number of frames times the average time to read a frame) should remain independent of $X$.
Therefore, the ratio for any $X$ should be $\RcompIO(X) = \tcomp(X)/\tIO(X) = X \RcompIO(X=1)$, i.e.,  $\RcompIO$ should just linearly scale with the workload factor $X$.
The measured $\RcompIO$ ratios of 11, 19, 27 for the increased computational workloads agreed with this theoretical analysis, as shown in Table \ref{tab:load-ratio}.

\begin{SCtable}[1.0][!htb]
\centering
\caption[Change in load-ratio with RMSD workload]{Change in $\RcompIO$ ratio with change in the RMSD workload $X$.
  The RMSD workload was artificially increased in order to examine the effect of compute to I/O ratio on the performance.
  The reported compute and I/O time were measured based on the serial version using one core.
  The theoretical $\RcompIO$ (see text) is provided for comparison.}
\label{tab:load-ratio}
\begin{tabular}{rrrrr}
  \toprule
  \bfseries\thead{Workload $X$} &  \bfseries\thead{$\tcomp$ (s)} &  \bfseries\thead{$\tIO$ (s)}
  & \multicolumn{2}{c}{\bfseries\thead{$\RcompIO$}}\\
  & & & \thead{measured} & \thead{theoretical}\\
  \midrule
    $1\times$   &   226 & 791 &  0.29 &   \\  
    $40\times$  &  8655 & 791 & 11   & 11\\    
    $70\times$  & 15148 & 791 & 19   & 20\\  
    $100\times$ & 21639 & 791 & 27   & 29\\  
  \bottomrule
\end{tabular}
\end{SCtable}

\begin{figure}[!htb]
  \centering
  \begin{subfigure} {.3\textwidth}
    \includegraphics[width=\linewidth]{figures/Compute_to_IO_ratio_on_performance_2d_v17.pdf}
    \caption{Speed-Up}
    \label{fig:S1_tcomp_tIO_effect}
  \end{subfigure}
  \hfill
  \begin{subfigure}{.3\textwidth}
    \includegraphics[width=\linewidth]{figures/Compute_to_IO_ratio_on_performance_2d_2_v17.pdf}
    \caption{Speed-Up}
    \label{fig:S2_tcomp_tIO_effect}
  \end{subfigure}
  \hfill
  \begin{subfigure}{.3\textwidth}
    \includegraphics[width=\linewidth]{figures/Compute_to_IO_ratio_on_performance_2d_3_v17.pdf}
    \caption{Efficiency}
    \label{fig:E_tcomp_tIO_effect}
  \end{subfigure}
  \caption{Effect of $\RcompIO$ ratio on performance of the RMSD task on \emph{SDSC Comet}. We tested performance for $\RcompIO$ ratios of 0.3, 11, 19, 27, which correspond to $1\times$ RMSD, $40\times$ RMSD, $70\times$ RMSD, and $100\times$ RMSD respectively.
    (a) Effect of $\RcompIO$ on the speed-up.
    (b) Change in speed-up with respect to $\RcompIO$ for different processor counts.
    (c) Change in the efficiency with respect to $\RcompIO$ for different processor counts.}
  \label{fig:tcomp_tIO_effect}
\end{figure}

We performed the experiments with increased workload to measure the effect of the $\RcompIO$ ratio (Eq.~\ref{eq:Compute-IO}) on performance (Figure~\ref{fig:tcomp_tIO_effect}).
The strong scaling performance as measured by the speed-up $S(N)$ improved with increasing $\RcompIO$ ratio (Figure \ref{fig:S1_tcomp_tIO_effect}).
The calculations consistently showed better scaling performance to larger numbers of cores for higher $\RcompIO$ ratios, e.g., $N=56$ cores for the $70\times$ RMSD task. 
The speed-up and efficiency approached their ideal value for each processor count with increasing $\RcompIO$ ratio (Figures \ref{fig:S2_tcomp_tIO_effect} and \ref{fig:E_tcomp_tIO_effect}).
Even for moderately compute-bound workloads, such as the $40\times$ and $70\times$ RMSD tasks, increasing the computational workload over I/O reduced the impact of stragglers even though they still contributed to large variations in timing across different ranks and thus irregular scaling.



%%% Local Variables:
%%% mode: latex
%%% TeX-master: "main"
%%% End:

% -*- mode: latex; mode: visual-line; fill-column: 9999; coding: utf-8 -*-

\section{Effect of \Rcompcomm on Performance}
\label{sec:tcomm}

In Section \ref{sec:I/O}, we improved scaling limitations due to read I/O by splitting the trajectory, but scaling remained far from ideal when MPI communication was used; somewhat surprisingly, using \emph{Global Arrays} lead to better scaling (see Section \ref{Global-Array}) because the effective communication cost was reduced \mknote{I really disagree we this sentence. I do not understand why the results of the GA is surprising for every one. I have tried to explain why GA lead to better perforance. What is not clear to you?}.
Although we were not able to identify the reason for the better performance of \emph{Global Arrays} (it still uses MPI as a communicator), the results motivated an analysis in terms of the communication costs.
In addition to the compute to I/O ratio \RcompIO discussed in Section \ref{sec:bound} we defined another performance parameter called the compute to communication ratio $\Rcompcomm$ (Eq.~\ref{eq:Compute-comm}).

We analyzed the data for variable workloads (see Section \ref{sec:bound}) in terms of the $\Rcompcomm$ ratio.
The performance clearly depended on the $\Rcompcomm$ ratio (Figure \ref{fig:tcom_tcomm_effect}).
Performance improved with increasing $\Rcompcomm$ ratios (Figure \ref{fig:tcomp_tcomm_ratio} and \ref{fig:S1_tcomp_tcomm_effect}) even if the communication time was larger (Figure \ref{fig:Comm_time_tcomp_tcomm_effect}).
Although we still observed stragglers due to communication at larger $\Rcompcomm$ ratios ($70\times$ RMSD and $100\times$ RMSD), their effect on performance remained modest because the overall performance was dominated by the compute load. 
Evidently, as long as overall performance is dominated by a component such as compute that scales well, then performance problems with components such as communication will be masked and overall acceptable performance can still be achieved (Figures \ref{fig:S1_tcomp_tcomm_effect} and \ref{fig:tcomp_tcomm_ratio}).

Communication was usually not problematic within one node because of the shared memory environment.
For less than 24 processes, i.e., a single compute node on \emph{SDSC Comet}, the scaling was good and $\Rcompcomm \gg 1$ for all RMSD loads (Figures \ref{fig:S1_tcomp_tcomm_effect} and \ref{fig:tcomp_tcomm_ratio}).
However, beyond a single compute node ($>$ 24 cores), scaling appeared to improve with increasing $\Rcompcomm$ ratio while the communication overhead decreased in importance (Figures \ref{fig:S1_tcomp_tcomm_effect} and \ref{fig:tcomp_tcomm_ratio}).

\begin{figure}[!htb]
  \centering
  \begin{subfigure} {.33\textwidth}
    \includegraphics[width=\linewidth]{figures/Compute_to_IO_ratio_on_performance_2d_v17.pdf}
    \caption{Speed-Up $S(N)$}
    \label{fig:S1_tcomp_tcomm_effect}
  \end{subfigure}
  \hfill
  \begin{subfigure}{.3\textwidth}
    \includegraphics[width=\linewidth]{figures/Compute_to_comm_ratio_on_performance_v17.pdf}
    \captionsetup{format=hang}
    \caption{Compute to communication ratio \Rcompcomm}
    \label{fig:tcomp_tcomm_ratio}
  \end{subfigure}
  \hfill
  \begin{subfigure}{.33\textwidth}
    \includegraphics[width=\linewidth]{figures/comm_comparison_different_RMSD_overload.pdf}
    \caption{Communication time \tcomm}
    \label{fig:Comm_time_tcomp_tcomm_effect}
  \end{subfigure}
  \caption{Effect of the ratio of compute to communication time \Rcompcomm on scaling performance on \emph{SDSC Comet}.
    (a) Scaling for different computational workloads. (Same as Figure~\protect\ref{fig:S1_tcomp_tIO_effect}.)
    (b) Change in \Rcompcomm with the number of processes $N$ for different workloads. 
    (c) Comparison of communication time for different RMSD workloads.
    Five repeats were performed to collect statistics and error bars show standard deviation with respect to mean.}
  \label{fig:tcom_tcomm_effect}
\end{figure}

%%% Local Variables:
%%% mode: latex
%%% TeX-master: t
%%% End:


%%% Local Variables:
%%% mode: latex
%%% TeX-master: "main"
%%% End:
